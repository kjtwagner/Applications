\documentclass{article}
\usepackage[utf8]{inputenc}
\usepackage[letterpaper, margin=1in]{geometry}
\thispagestyle{empty}

\begin{document}

In eighth grade, a teacher I respected voiced that women were more suited to studying English than math and told me I should work on my writing rather than spending so much time thinking about my alegra class. Though even at the time I knew this sentiment was ludicrous, I have felt driven to both perform at my very best academically. It also compelled me to support other women who hope to pursue their STEM interests.

During my first two years of graduate school, I worked on a search for continuous gravitational waves from Scorpius X-1 with Dr. John Whelan. Scorpius X-1 is considered a high priority candidate by the continuous waves (CW) group in the LIGO-Virgo-KAGRA (LVK) collaboration, but parameter uncertainty combined with the weak signal strength of CWs make searches difficult and computationally expensive. I spent my time as a Master’s student improving our cross-correlation code to perform searches for CWs more efficiently, so that more sensitive searches could be performed on our fixed computing budget. I calculated a parameter space transformation and added it to our search pipeline, which uses both python and C. I performed tests to calculate that my changes improved the computing cost by a factor of 3. This work was published in Classical and Quantum Gravity in January 2022 and became the foundation of my Master’s thesis, which I defended in July 2022. My work enabled our group to perform the most sensitive search to date for CWs from Scorpius X-1, and the results of this work were published in a paper on behalf of the LVK collaboration. I presented this work on numerous occasions at LVK collaboration meetings as well as at the 14th Edoardo Amaldi Conference on Gravitational Waves. Now, my research involves eccentric parameter estimation of low-mass binary black hole mergers, another key topic in the LVK.

However, while research accomplishments are an important part of my graduate student career, I highly value opportunities to engage with young students from underrepresented groups in astrophysics. According to the 2019 AIP report, only 21\% of physics Bachelor’s degrees were awarded to women. My personal experience as often the only woman in my physics and astronomy classes is paradigmatic of this low statistical representation in a field that is already difficult enough. My goal is to support and encourage young women considering astrophysics. I currently work with high school sophomore Haylli Yunga from Ossining High School on a long term project studying eccentricity and precession in binary black hole mergers. The project is carefully defined such that she can make a significant impact related to current work in the field that can be accomplished on a two-year timescale. To help Haylli, I have created a timeline for the project with a plan for how she can progress through the stages of research as she learns. I have gathered resources for her to begin learning to code in python and papers for her to learn the appropriate background material such that she can successfully execute the project. By her senior year of high school, Haylli will have completed her research project, presented her findings on multiple occasions, and be able to produce some level of publication that will supplement her college applications.I am thrilled to guide and mentor Haylli as she works hard to achieve her goals.

I am also collaborating with fellow AST student Marko Ristić to develop an astronomy outreach curriculum geared towards high school students. There are not many opportunities for high school students to actively participate in interactive astronomy research experience, but having some familiarity with astronomy and the opportunities available in the field can inform students as they make important decisions about college and their future. Executing this involves collecting and reading reference material, writing a thorough explanation or lecture of complex topics in astronomy to be understood at the high school level, and writing and developing code modules for students to use interactively that enhances their understanding of the lecture material. The goal is to invite students to participate beginning this summer, which will help grow the field and introduce a broader audience to the possible opportunities. 

Active teaching and mentoring are key to encouraging young students to get involved in the field. However, it can often be just as essential to provide simple support for students as they progress through their academic journeys. I have now been a peer mentor for the AST graduate program for two years, providing encouragement and advice to fellow graduate students as they navigate the difficulties of their first year. I meet bimonthly with my students in an informal setting for the entirety of their first academic year. I have mentored two students per year, 3/4 being other women in the program. I also continue to remain involved with high school students at my alma mater, primarily in the AP Computer Science class and the Girls Who Code club. I check in periodically to assist students with class or club projects, provide college advice, and occasionally guest lecture about topics in astrophysics relevant to their course work. Through this partnership I hope that students come to understand that the pursuit of STEM fields, particularly computer science, physics, and astronomy, is valuable and achievable.

Conclusion


\end{document}
