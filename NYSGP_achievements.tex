\documentclass{article}
\usepackage[utf8]{inputenc}
\usepackage[letterpaper, margin=1in]{geometry}

\begin{document}

``A one-page statement outlining notable achievements (e.g., publications, presentations, prizes,
other forms of recognition) and community engagement (e.g., outreach or service activities) and
any other relevant information (e.g., leadership roles).''

1. Notable achievements

During my first two years of graduate school, I worked on a search for continuous gravitational waves from Scorpius X-1 with Dr. John Whelan. Scorpius X-1 is considered a high priority candidate by the continuous waves (CW) group in the LIGO-Virgo-KAGRA (LVK) collaboration, but parameter uncertainty combined with CW weak signal strength make searches difficult and computationally expensive. I spent my time as a Master’s student improving our cross-correlation code to perform searches for CWs more efficiently, so that more sensitive searches could be performed on our fixed computing budget. I calculated a parameter space transformation and added it to our search pipeline, which uses both python and C. I performed tests to calculate that my changes improved the computing cost by a factor of 3. This work was published in Classical and Quantum Gravity in January 2022 and became the foundation of my Master’s thesis, which I defended in July 2022. My work enabled our group to perform the most sensitive search to date for CWs from Scorpius X-1, and the results of this work were published in a paper on behalf of the LVK collaboration. I presented this work on numerous occasions at LVK collaboration meetings as well as at the 14th Edoardo Amaldi Conference on Gravitational Waves.

2. Community engagement

However, while research accomplishments are an important part of my graduate student career, I highly value opportunities to engage with young students from underrepresented groups in astrophysics. According to the 2019 AIP report, only 21\% of physics Bachelor’s degrees were awarded to women. My personal experience as often the only woman in my physics and astronomy classes is paradigmatic of this low statistical representation in a field that is already difficult enough. My goal is to support and encourage young women considering astrophysics. I currently work with high school sophomore Haylli Yunga from Ossining High School on a long term project studying eccentricity and precession in binary black hole mergers. The project is carefully defined such that she can make a significant impact related to current work in the field that can be accomplished on a two-year timescale. 
I am also collaborating with fellow AST student Marko Ristić to develop an astronomy outreach curriculum geared towards high school students. There are not many opportunities for high school students to actively participate in interactive astronomy research experience, but having some familiarity with astronomy and the opportunities available in the field can inform students as they make important decisions about college and their future. Executing this involves collecting and reading reference material, writing a thorough explanation or lecture of complex topics in astronomy to be understood at the high school level, and writing and developing code modules for students to use interactively that enhances their understanding of the lecture material. The goal is to invite students to participate beginning this summer, which will help grow the field and introduce a broader audience to the possible opportunities. 


Action: Creating a timeline for the project, and a plan for how to progress through the stages of the project. Gathering resources for learning the required skills, including python and basic differential equations. Mentoring and teaching \ldots

Results: By her senior year of high school, Haylli will have completed a research project exploring … and have presented her findings on multiple occasions, as well as have some level of publication describing her results.


- Continue to be involved with C-NS AP CS class and GWC club

3. Leadership

- AST Graduate student peer mentor, 2 years


\end{document}
