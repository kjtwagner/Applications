\documentclass{article}
\usepackage[utf8]{inputenc}
\usepackage[letterpaper, margin=1in]{geometry}

\begin{document}

``A one-page statement outlining notable achievements (e.g., publications, presentations, prizes,
other forms of recognition) and community engagement (e.g., outreach or service activities) and
any other relevant information (e.g., leadership roles).''

1. Notable achievements

- Lattice paper describing factor 3 computing cost due to code changes. Became Master's thesis
Challenge: 
- Routinely present at LIGO-Virgo-KAGRA collaboration meetings

2. Community engagement

- Working with high school sophomore Haylli Yunga on longer term research project
Challenge: Haylli is a sophomore HS student interested in astronomy, and sought out a project that she can contribute to over the course of her high school career.

Action: Deciding a project topic that is relevant to current work in the field that can be accomplished by a high school student on a two-year timescale. Creating a timeline for the project, and a plan for how to progress through the stages of the project. Gathering resources for learning the required skills, including python and basic differential equations. Mentoring and teaching \ldots

Results: By her senior year of high school, Haylli will have completed a research project exploring … and have presented her findings on multiple occasions, as well as have some level of publication describing her results.

- Developing high school outreach curriculum with Marko

Challenge: there are not many opportunities for high school students to actively participate in interactive/hands-on astronomy research experience, but having some familiarity with astronomy and the opportunities available in the field can inform students decisions as they decide whether/where/what to apply for college. This can help grow the field and introduce a broader audience to the possible opportunities. NEED RESOURCES TO DO THIS

Action: Developing a curriculum where HS students can experience what its like to perform astronomy research in three categories - observational astronomy, theoretical astronomy/astrophysics, and astronomical instrumentation. Involves collecting and reading reference material, writing a thorough explanation/lecture of complex topics in astronomy to be understood at the high school level, writing and developing code modules for students to use interactively that enhances/furthers their understanding of the lecture material

Results:

- Continue to be involved with C-NS AP CS class and GWC club

3. Leadership

- AST Graduate student peer mentor, 2 years


\end{document}
